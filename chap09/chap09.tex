\documentclass[a4paper,12pt]{article}

\usepackage{xeCJK}
\setCJKmainfont{Songti SC}
\usepackage{fullpage}
\usepackage{graphicx}
\usepackage{setspace}
\onehalfspacing

\begin{document}
\title{程序设计语言概论\ 第9章作业}
\author{赵睿哲\\1200012778}
\maketitle

\section{问题研讨}

\subsection{9a}

\begin{enumerate}
\item \textbf{子程序(调用)和宏(展开)的区别?}作用阶段不同,宏主要在预处理阶段进行展开,而子程序则是在运行时进行调用;
\item \textbf{子程序为什么需要局部存储变量?}为了与全局环境隔离开,否则非常不方便,而且无法实现递归;
\item \textbf{传值、传引用、传名的本质区别是什么?}传值将实参的值复制到形参,传引用是令形参通过实参的路径访问实参,传名则通过简单的文本替换。
\item\textbf{ 按值-结果传递与按引用传递有什么区别?}在输入过程中,值-结果在局部存储保存值、引用保存指针;在输出过程中,值-结果将局部存储的值复制到实参中,引用则直接修改指针对应的内存区域中的值。
\item \textbf{如果不允许使用非局部变量,也会产生别名吗?}会,只要有引用机制,就一定会产生别名的问题。
\end{enumerate}

\subsection{9b}

\begin{enumerate}
\item \textbf{在实现参数传递时,一般是在被调用子程序中为参数分配存储空间,可不可以在调用子程序中为参数分配存储空间?} 可以,但是实现起来比较复杂,一般来说简单的运行时堆栈可以直接通过修改栈顶来释放子程序调用使用的存储空间,但是如果参数可以在调用子程序中分配的话,调用返回之后需要更多的步骤来释放这些存储。
\item \textbf{在数组或记录做为参数时,可以只传递参数的地址吗?如何实现结构化数据的参数传递?}1) 取决于使用场景,对于需要数组长度的情况,数组的长度信息也需要传递;2) 根据结构化数据的类型,分配存储空间,并把结构化数据的内容原样拷贝到新分配的存储空间中。
\item \textbf{如何检测重载冲突?重载和多态的区别?}通过参数类型即可区分不同的重载函数;重载只是令一类函数可以使用一个名字处理多种参数类型,而多态是与面向对象相关的,必须存在类和对象之间的继承关系才行。
\item \textbf{通用子程序是子程序吗?如何才能调用吗?}是子程序的模板,在具体调用的地方会被实例化为具体的子程序;与一般的子程序调用基本类似,不过需要增加对类型的指定。
\item \textbf{协同程序和子程序的本质区别是什么?可不可以用子程序来模拟实现协同程序?}本质区别是子程序在返回到调用者之后就被销毁,无法回到原来的环境;可以模拟,将协同程序中以resume划分的代码块都变为独立的子程序。
\end{enumerate}

\section{作业}

\subsection{9.2}

\subsection{9.5}

\subsection{9.7}

\end{document}